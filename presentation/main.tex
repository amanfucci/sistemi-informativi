\documentclass[english, aspectratio=169, final]{beamer}
\usepackage[english]{babel}
\usepackage{ragged2e}
\usepackage{ifthen}
\usepackage{graphicx}
\usepackage{booktabs}
\usepackage{tikz}
\usetikzlibrary{
  backgrounds,
  arrows,
  arrows.meta,
  shapes,
  shapes.geometric,
  shapes.misc,
  decorations.pathmorphing,
  decorations.markings
}
\usepackage{tikzit}
\usepackage{multicol}
\usepackage{amsmath}
\usepackage{bbm}
\usepackage[ruled,noend,vlined]{algorithm2e}
\usepackage{mathtools}
\usepackage{xcolor}
\usepackage{doi}
\usepackage{pgfplots}
\usepackage{stackengine}
\stackMath{}
\pgfplotsset{compat=1.9}
\usepgfplotslibrary{external}
\tikzexternalize[prefix=./build_tikz/]
\tikzset{external/only named=true}
\input{../graphs.tikzstyles}
\input{../graphs.tikzdefs}
\apptocmd{\frame}{}{\justifying}{}
\newcommand{\itemj}{\item \justifying}
\newcommand{\reaches}{\rightsquigarrow}
\renewcommand{\path}{,\ldots,}
\newcommand{\ureaches}{\leftrightsquigarrow}
\newcommand{\reached}{\leftsquigarrow}
\renewcommand{\implies}{\Rightarrow}
\newcommand{\impliesq}{\stackrel{?}{\implies}}
\newcommand{\implied}{\Leftarrow}
\newcommand{\impliedq}{\stackrel{?}{\implied}}
\renewcommand{\iff}{\Leftrightarrow}
\newcommand{\iffq}{\stackrel{?}{\iff}}
\newcommand{\iffdef}{\stackrel{\text{def}}{\iff}}
\renewcommand{\to}{\rightarrow}
\newcommand{\toq}{\stackrel{?}{\to}}
\newcommand{\from}{\leftarrow}
\newcommand{\fromq}{\stackrel{?}{\from}}
\newcommand{\eps}{\varepsilon}
\newcommand{\eqq}{\stackrel{?}{=}}
\newcommand{\inq}{\stackrel{?}{\in}}
\newcommand{\eqd}{\vcentcolon=}
\newcommand{\pluseq}{\mathrel{+}=}
\newcommand{\asteq}{\mathrel{*}=}
\newcommand{\minuseq}{\mathrel{-}=}
\newcommand{\expm}{e\!-\!}
\newcommand{\expp}{e\!+\!}
\newcommand{\impliesnotimplied}{\stackanchor{\implies}{\not\implied}}
\newcommand{\impliednotimplies}{\stackanchor{\not\implies}{\implied}}
\newcommand{\loc}{\text{loc}}
\newcommand{\glob}{\text{glob}}
\DeclareMathOperator{\adj}{adj}
\DeclareMathOperator{\adjin}{adj_{in}}
\DeclareMathOperator{\adjout}{adj_{out}}
\DeclareMathOperator{\degin}{deg_{in}}
\DeclareMathOperator{\degout}{deg_{out}}
\DeclareMathOperator{\Bin}{Bin}
\DeclareMathOperator{\Ber}{Ber}
\DeclareMathOperator{\Uni}{Uni}
\DeclareMathOperator{\enqueue}{enqueue}
\DeclareMathOperator{\dequeue}{dequeue}
\DeclareMathOperator{\size}{size}
\DeclareMathOperator{\push}{push}
\DeclareMathOperator{\pop}{pop}
\DeclareMathOperator{\Init}{Init}
\DeclareMathOperator{\Decrease}{Decrease}
\DeclareMathOperator{\Increase}{Increase}
\DeclareMathOperator{\Acc}{Acc}
\DeclareMathOperator{\Distance}{Distance}
\DeclareMathOperator{\polylog}{polylog}

\newcommand{\kwangles}[1]{
  $\mathord{\langle}\text{#1}\mathord{\rangle}$%
}

\graphicspath{ {../images/} }

% Style
\definecolor{red}{rgb}{0.796, 0.055, 0.173} % primary
\definecolor{grey}{rgb}{0.3686, 0.5255, 0.6235} % secondary
\setbeamercolor{palette primary}{bg=red,fg=white}
\setbeamercolor{palette secondary}{bg=red,fg=white}
\setbeamercolor{palette tertiary}{bg=red,fg=white}
\setbeamercolor{palette quaternary}{bg=red,fg=white}
\setbeamercolor{structure}{fg=red}
\setbeamercolor{bibliography item}{fg=red}
\setbeamercolor{bibliography entry author}{fg=black}
\setbeamercolor{bibliography entry title}{fg=black}
\setbeamercolor{bibliography entry location}{fg=black}
\setbeamercolor{bibliography entry note}{fg=black}
\bibliographystyle{plain}
\setbeamertemplate{caption}[numbered]
\setbeamertemplate{navigation symbols}{}
\setbeamertemplate{bibliography item}{[\theenumiv]}
\setbeamertemplate{section in toc}[circle]
\setbeamertemplate{subsection in toc}[ball unnumbered]
\setbeamertemplate{headline}
{
    \leavevmode
    \hbox{
    \begin{beamercolorbox}[wd=\paperwidth,ht=2.5ex,dp=1.125ex]{palette quaternary}
    \insertsectionnavigationhorizontal{\paperwidth}{}{\hskip0pt plus1filll}
    \end{beamercolorbox}
    }
}
\setbeamertemplate{footline}
{
    \leavevmode
    \hbox{
    \begin{beamercolorbox}[wd=.33\paperwidth,ht=2.6ex,dp=1ex,center]{palette quaternary}
    \usebeamerfont{author in head/foot}\insertshortauthor\hspace*{1ex}
    \end{beamercolorbox}
    \begin{beamercolorbox}[wd=.33\paperwidth,ht=2.6ex,dp=1ex,center]{palette quaternary}
    \usebeamerfont{institute in head/foot}\insertshortinstitute\
    \end{beamercolorbox}
    \begin{beamercolorbox}[wd=.33\paperwidth,ht=2.6ex,dp=1ex,center]{palette quaternary}
    \insertframenumber{} / \inserttotalframenumber\
    \end{beamercolorbox}}
    \vskip0pt
}
\newcommand{\customToC}[2]
{
    \begin{frame}[noframenumbering]{Overview}
        \tableofcontents[#1,#2]
    \end{frame}
}
\newcommand{\sectiontitle}{}
\newcommand{\subsectiontitle}{}
\setlength{\tabcolsep}{2pt}

\title{Progetto Sistemi Informativi}
\subtitle{Implementazione Dashboard ROLAP}
\author{Alessandro Manfucci}
\date{09 Febbraio 2025}
\institute{Università degli studi di Trento}
% \titlegraphic{\includegraphics[width=4cm]{marchio_unitrento_colore_it_202002.eps}}

\begin{document}
\SetKw{KwBreak}{break}
\SetKw{KwNew}{new}
\SetKwData{TyQueue}{Queue}
\SetKwData{TyPriorityQueue}{PriorityQueue}
\SetKwData{TyStack}{Stack}
\SetKwData{TySet}{Set}
\SetKwData{tyint}{int}
\SetKwData{tyfloat}{float}
\SetKwData{tybool}{bool}
\small

% Title Frame
\begin{frame}[noframenumbering]
    \centering
    \vspace{1cm}
    {\color{red} \Large \inserttitle\par}
    \vspace{0.25cm}
    {\color{red} \normalsize \insertsubtitle\par}
    \vspace{1cm}
    {\normalsize \insertauthor\par}
    \vspace{1cm}
    {\small \insertdate\par}
    \vspace{0.5cm}
    {\inserttitlegraphic\par}
    \vspace{1cm}
\end{frame}

% TOC Frame
\customToC{}{}
\setbeamercovered{invisible}

% Section
\renewcommand{\sectiontitle}{Strumenti}
\section[Strument]{\sectiontitle}
% \customToC{currentsection,hideothersubsections}{}

% Frame
\begin{frame}[t]{\sectiontitle}
    \begin{multicols}{2}
        \begin{itemize}
            \itemj{}SDMX
            \begin{itemize}
                \itemj{}Dataflows: Facts
                \itemj{}Data Structure Definitions: Fact Schemas
                \itemj{}Codelists: Enum domains (misure, dimensioni)
            \end{itemize}
            \itemj{}Apache Superset: ROLAP
            \begin{itemize}
                \itemj{}Pivot righe-colonne
                \itemj{}Slice, Dice come filtri
                \itemj{}Drill-Down/Roll-up richiedono Pre-Aggregazione
            \end{itemize}
        \end{itemize}
        \columnbreak{}
        \begin{center}
            \includegraphics[width=0.3\textwidth]{tools.png}
        \end{center}
    \end{multicols}
\end{frame}

% Section
\renewcommand{\sectiontitle}{Dati}
\section[Dati]{\sectiontitle}
% \customToC{currentsection,hideothersubsections}{}

% Frame
\begin{frame}[t]{Microdati vs Macrodati}
    \begin{center}
        \includegraphics[width=0.8\textwidth]{micro-to-macro-dfm.png}
    \end{center}
\end{frame}

% Frame
\begin{frame}[t]{Microdati vs Macrodati}
    \begin{center}
        \includegraphics[width=0.8\textwidth]{micro-to-macro-tabelle.png}
    \end{center}
\end{frame}

% Subsection
\renewcommand{\subsectiontitle}{Cybersecurity Policies}
\subsection[Policies]{\subsectiontitle}
% \customToC{currentsection,hideothersubsections}{}

% Frame
\begin{frame}[t]{\subsectiontitle}
    \begin{multicols}{2}
        \begin{itemize}
            \itemj{}Le dimensioni sono:
            \begin{itemize}
                \itemj{}anno
                \itemj{}nazione
                \itemj{}grandezza dell'azienda
            \end{itemize}
            \itemj{}Per ciascun indicatore, la misura è una stima percentuale della quota di imprese della popolazione che soddisfano la condizione dell'indicatore
        \end{itemize}
        \columnbreak{}
        \begin{center}
            \includegraphics[width=0.4\textwidth]{isoc_cisce_ra.png}
        \end{center}
    \end{multicols}
\end{frame}

% Frame
\begin{frame}[t]{\subsectiontitle}
    \begin{multicols}{2}
        Granularità indicatori
        \begin{itemize}
            \itemj{}E\_SECM:~Misure di sicurezza
            \itemj{}E\_ICTSEC3:~ICT Interno o Esterno
            \itemj{}E\_SECPREV:~Quanto è recente l'aggiornamento delle misure di sicurezza
            \itemj{}E\_SEC\_AW:~Come il personale è istruito alla sicurezza
        \end{itemize}
        \columnbreak{}
        \begin{center}
            \includegraphics[width=0.4\textwidth]{isoc_cisce_ra_indic.png}
        \end{center}
    \end{multicols}
\end{frame}

% Subsection
\renewcommand{\subsectiontitle}{Cybersecurity Incidents}
\subsection[Incidents]{\subsectiontitle}
% \customToC{currentsection,hideothersubsections}{}

% Frame
\begin{frame}[t]{\subsectiontitle}
    \begin{multicols}{2}
        \begin{itemize}
            \itemj{}Le dimensioni sono:
            \begin{itemize}
                \itemj{}anno
                \itemj{}nazione
                \itemj{}grandezza dell'azienda
            \end{itemize}
            \itemj{}Per ciascun indicatore, la misura è una stima percentuale della quota di imprese della popolazione che soddisfano la condizione dell'indicatore
       \end{itemize}
        \columnbreak{}
            \begin{center}
            \includegraphics[width=0.4\textwidth]{isoc_cisce_ic.png}
        \end{center}
    \end{multicols}
\end{frame}

% Frame
\begin{frame}[t]{\subsectiontitle}
    \begin{multicols}{2}
        Granularità indicatori
        \begin{itemize}
            \itemj{}E\_SEC2I:~Incidenti legati alla sicurezza
            \begin{itemize}
                \itemj{}USV:~Interruzione del servizio
                \itemj{}DCD:~Perdita di dati
                \itemj{}CNF:~Divulgazione di dati sensibili
            \end{itemize}
            \itemj{}E\_SECINS:~Incidenti coperti da assicurazione
        \end{itemize}
        \columnbreak{}
        \begin{center}
            \includegraphics[width=0.4\textwidth]{isoc_cisce_ic_indic.png}
        \end{center}
    \end{multicols}
\end{frame}

% Subsection
\renewcommand{\subsectiontitle}{Privacy Awareness}
\subsection[Awareness]{\subsectiontitle}
% \customToC{currentsection,hideothersubsections}{}

% Frame
\begin{frame}[t]{\subsectiontitle}
    \begin{multicols}{2}
        \begin{itemize}
            \itemj{}Le dimensioni sono:
            \begin{itemize}
                \itemj{}anno
                \itemj{}nazione
                \itemj{}classe
            \end{itemize}
            \itemj{}Per ciascun indicatore, la misura è una stima percentuale della quota di individui della popolazione che soddisfano la condizione dell'indicatore
        \end{itemize}
        \columnbreak{}
        \begin{center}
            \includegraphics[width=0.4\textwidth]{isoc_cisci_prv20.png}
        \end{center}
    \end{multicols}
\end{frame}

% Frame
\begin{frame}[t]{\subsectiontitle}
    \begin{multicols}{2}
        Granularità indicatori
        \begin{itemize}
            \itemj{}I\_MAPS:~L'individuo gestisce l'accesso ai dati personali
            \begin{itemize}
                \itemj{}RPS:~Ha letto la privacy policy
                \itemj{}RRGL:~Ha ristretto l'accesso alla posizione
                \itemj{}LAP:~Ha limitato la condivisione di dati su social network o cloud
                \itemj{}RAAD:~Ha rifiuto l'uso di cookie ai fini pubblicitari
                \itemj{}CWSC:~Ha controllato che il sito fosse sicuro
            \end{itemize}
        \end{itemize}
        \columnbreak{}
        \begin{center}
            \includegraphics[width=0.4\textwidth]{isoc_cisci_prv20_indic.png}
        \end{center}
    \end{multicols}
\end{frame}

% Subsection
\renewcommand{\subsectiontitle}{ICT GVA}
\subsection[ICT GVA]{\subsectiontitle}
% \customToC{currentsection,hideothersubsections}{}

% Frame
\begin{frame}[t]{\subsectiontitle}
    \begin{multicols}{2}
        \begin{itemize}
            \itemj{}Le dimensioni sono:
            \begin{itemize}
                \itemj{}anno
                \itemj{}nazione
                \itemj{}classe economica nace
            \end{itemize}
            \itemj{}La misura è la percentuale del settore ICT sul GVA totale delle aziende nella popolazione
            \begin{itemize}
                \itemj{}Fino al 2020: GVA factor cost
                \itemj{}Dal 2021: GVA basic prices
            \end{itemize}
        \end{itemize}
        \columnbreak{}
        \begin{center}
            \includegraphics[width=0.4\textwidth]{tin00074.png}
        \end{center}
    \end{multicols}
\end{frame}

% Subsection
\renewcommand{\subsectiontitle}{ICT Personnel}
\subsection[ICT Personnel]{\subsectiontitle}
% \customToC{currentsection,hideothersubsections}{}

% Frame
\begin{frame}[t]{\subsectiontitle}
    \begin{multicols}{2}
        \begin{itemize}
            \itemj{}Le dimensioni sono:
            \begin{itemize}
                \itemj{}anno
                \itemj{}nazione
                \itemj{}classe economica nace
            \end{itemize}
            \itemj{}La misura è la percentuale del personale ICT rispetto al personale totale delle aziende nella popolazione
            \itemj{}Il personale è classificato sul settore, non sull'occupazione
        \end{itemize}
        \columnbreak{}
        \begin{center}
            \includegraphics[width=0.4\textwidth]{tin00085.png}
        \end{center}
    \end{multicols}
\end{frame}

% Section
\renewcommand{\sectiontitle}{Dashboard}
\section[Dashboard]{\sectiontitle}
% \customToC{currentsection,hideothersubsections}{}

% Frame
\begin{frame}[t]{\sectiontitle}
    \begin{center}
        \includegraphics[width=1\textwidth]{dashboard1.jpg}
    \end{center}
\end{frame}

% Frame
\begin{frame}[t]{\sectiontitle}
    \begin{center}
        \includegraphics[width=1\textwidth]{dashboard2.jpg}
    \end{center}
\end{frame}

% Frame
\begin{frame}[t]{\sectiontitle}
    \begin{center}
        \includegraphics[width=1\textwidth]{dashboard3.jpg}
    \end{center}
\end{frame}

% Frame
\begin{frame}[t]{\sectiontitle}
    \begin{center}
        \includegraphics[width=1\textwidth]{dashboard4.jpg}
    \end{center}
\end{frame}

% Frame
\begin{frame}[t]{Thanks!}
\end{frame}

% Bibliography
% \section[Refs]{References}
% \begin{frame}[t, allowframebreaks, noframenumbering]
% \frametitle{References}
% \bibliography{biblio}
% \end{frame}

\end{document}